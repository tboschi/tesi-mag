%%%%%%%%%%%%%%%%%%%%%%%%%%%%%%%		CHAP 6		%%%%%%%%%%%%%%%%%%%%%%%%%%%%%%%

\chapter[Measurement]{Data analysis}
\label{cha:5}
\textcolor{blue}{very scarce section, for now.}


For the analysis, 6 runs were selected:

\begin{center}
  \small
  \begin{tabular}{clr}
    \toprule
    \textbf{Run}	& Description	& \textbf{Data size}		\\
    			&		&	(kB)			\\
    \midrule
    93		& beam off, trigger random	& $ 65027961	$	\\
    94		& beam off, trigger random	& $ 41102486	$	\\
    120		& after card synchronisation	& $ 19887893	$	\\
    121		& after card synchronisation	& $ 6735270	$	\\
    145		& PMT mounted on NCV		& $ 37204525	$	\\
    146		& PMT mounted on NCV		& $ 39050909	$	\\
    \bottomrule
  \end{tabular}
\end{center}


\section{Cosmic background data}
With the beam off, only cosmic rays and background leave a trace in the detector.
Therefore runs 93 and 94's sources are useful to characterise the background.

\section{Threshold dependance}

Three parameters of the Data Analysis software have been tuned so as to study the feedback %
of the rejection method.
The parameters are:
\begin{enumerate}
  \item Voltage threshold;
  \item Number of PMTs fired;
  \item Time rejection window.
\end{enumerate}

\subsection{Voltage}
\subsection{PMTs fired}
\subsection{Rejection window}

\section{Michel decay}
I've seen it, it could be a nice plot to place here. 

\section{Neutron yield}
No luck yet.
really not sure about this section.

\section{First MRD data}
TDC, could do some dummy analysis.
