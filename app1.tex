%%%%%%%%%%%%%%%%%%%%%%%%%%%%%%%		APP		%%%%%%%%%%%%%%%%%%%%%%%%%%%%%%%

\chapter{Booster Neutrino Beam}
\label{app:A}

Arxiv 1311.5958v1.
Fermilab has two major neutrino beamlines: the Neutrino Main Injector (NuMI) and %
the Booster Neutrino Beam (BNB).
The energy range of these two neutrino sources on-axis is in the GeV range, %
which is too high to satisfy the condition for dominance of coherent scattering. 
The far-off-axis (> $45^\circ$) of the BNB produces well defined neutrinos with %
energies below 50 MeV.
The BNB source has substantial advantages over the NuMI beam source owing to suppressed %
kaon production from the relatively low energy 8 GeV proton beam on the target~ref. 
Therefore, pion decay and subsequent muon decay processes are the dominant sources of neutrinos. 
At the far-off-axis area, the detector can be placed close enough to the target to gain a %
large increase in neutrino flux due to the larger solid angle acceptance.

\section{BNB details}
An initial study using the existing BNB MC has confirmed that this approach is promising.
The Fermilab Booster is a 474-meter-circumference synchrotron operating at 15 Hz. 
Protons from the Fermilab LINAC are injected at 400 MeV and accelerated to 8 GeV kinetic energy. 
The structure of the beam is a series of 81 proton bunches each with a 2 ns width and 19 ns apart. 
The maximum average repetition rate for proton delivery to the BNB is 5 Hz and 5 x 10 12 protons %
per pulse. 
The repetition limit is set by the horn design and its power supply. 
The target is made of beryllium divided in seven cylindrical sections in a total of \np{71.1}~cm %
in length and 0.51 cm in radius.
In order to minimize up-stream proton interactions, the vacuum of the beam pipe extends to %
about 152 cm upstream of the target.
The horn is an aluminum alloy toroidal electromagnet with operating values of 174 kA and %
maximum field value of 1.5 Tesla. 
A concrete collimator is located downstream of the target and guides the beam into the decay region.
The air-filled cylindrical decay region extends for 45 meters. 
The beam stop is made of steel and concrete. 

PHYSICAL REVIEW D 79, 072002 (2009).
The Fermi National Accelerator Laboratory (FNAL) booster is a 474-meter-circumference %
synchrotron operating at 15 Hz. 
Protons from the Fermilab LINAC are injected at 400 MeV and accelerated to 8 GeV kinetic %
energy (8:89 GeV=c momentum).
he booster has a harmonic number of 84, of which 81 buckets are filled. 
Thebeam is extracted into the BNB using a fast-rising kicker that extracts all of the particles %
in a single turn.
The resulting structure is a series of 81 bunches of protons each 2 ns wide and 19 ns apart.
Upon leaving the booster, the proton beam is transported through a lattice of focusing %
and defocusing quadrupole (FODO) and dipole magnets.
A switch magnet steers the beam to the main injector or to the BNB. 
The BNB is also a FODO that terminates with a triplet that focuses the beamon the target. 
The design and measured optics of BNB are in agreement [7,8].
The maximum allowable average repetition rate for delivery of protons to the BNB is %
5 Hz (with a maximumof 11 pulses in a row at 15 Hz) and 5 x 10to12 protons-per-pulse. 
The 5 Hz limit is set by the design of the horn (described below) and its power supply.

The target consists of seven identical cylindrical slugs of beryllium arranged to produce a %
cylinder 71.1 cm long and 0.51 cm in radius. 
The target is contained within a beryllium sleeve 0.9 cm thick with an inner radius of 1.37 cm.
Each target slug is supported within the sleeve by three ``fins'' (also beryllium) %
which extend radially out from the target to the sleeve. 
The volume of air within the sleeve is circulated to provide cooling for the target when the beam %
line is in operation. 
The target and associated assembly are shown in Fig.~2, where the top figure shows an exploded %
view of the various components (with the downstream end of the target on the right), %
and the bottom shows the components in assembled form. 
The choice of beryllium as the target material was motivated by residual radioactivity issues %
in the event that the target assembly needed to be replaced, as well as energy loss %
considerations that allow the air-cooling system to be sufficient.
Upstream of the target, the primary proton beam is monitored using four systems: %
two toroids measuring its intensity (protons-per-pulse), beam position monitors %
(BPM) and a multiwire chamber determining the beam width and position, %
and a resistive wall monitor (RWM) measuring both the time and intensity of the beam spills.
The vacuum of the beam pipe extends to about 5 feet upstream of the target, %
minimizing upstream proton interactions.
The toroids are continuously calibrated at 5 Hz with their absolute calibrations verified %
twice a year.
The calibrations have shown minimal deviation ( < 0:5\%).
The proton flux measured in the two toroids agree to within 2\%, compatible with the %
expected systematic uncertainties.
The BPMs are split-plate devices that measure the difference of charge induced on two plates. 
By measuring the change in beam position at several locations without intervening optics, %
the BPMs are found to be accurate to 0.1 mm (standard deviation). 
The multiwire is a wire chamber with 48 horizontal and 48 vertical wires and 0.5 mm pitch. 
The profile of the beam is measured using the secondary emission induced by the beam on the wires.
The RWM is located upstream of the target to monitor the time and intensity of the proton %
pulses prior to striking the target. 
While the data from the RWM did not directly entere analysis, it allowed many useful cross %
checks, such as those shown in Fig. 3.
The left figure shows a comparison of the production times of neutrinos observed %
in the MiniBooNE detector estimated based on the vertex and time reconstructed by the detector %
and subtracting the time-of-flight. 
This time is then compared to the nearest bucket as measured by the RWM. 
The distribution indicates that neutrino events can be matched not only to pulses %
from the booster, but to a specific bucket within the pulse.
The tails of the distribution result from the resolution of the vertex reconstruction %
of the neutrino event in the detector, which is needed to determine the time of the %
event and correct for the time-of-flight.
The right plot demonstrates the synchronization of the horn pulse (described in Sec. II C) %
with the delivery of the beam as measured by the RWM.

The horn current pulse is approximately a half-sinusoid of amplitude 174 kA, 143 s long, %
synchronized to each beam spill. 

The typical beam alignment and divergence measured by the beam position monitors located %
near the target are within 1 mm and 1 mrad of the nominal target center and axis direction, %
respectively; the typical beam focusing on target measured by beam profile monitors is %
of the order of 1–2 mm [root mean square(RMS)] in both the horizontal and vertical directions.
These parameters are well within the experiment requirements. 
The number of protons delivered to the BNB target is measured for each proton %
batch using two toroids located near the target along the beam line. 
The toroid calibration, performed on a pulse-by-pulse basis, provides a measurement %
of the number of protons to BNB with a 2\,\% accuracy.
Primary protons from the 8 GeV beam line strike a thick beryllium target located %
in the BNB target hall.

The SciBooNE hall is on axis from the Booster Neutrino Beam (BNB) at 8 m below the surface.
The BNB impinges 8.89 GeV/c protons from the booster on a beryllium target, with %
4 x10to12 delivered in a narrow spill of approximately 1.6 micros at a frequency of 7.5 Hz. 
The projected number of protons incident on the target (POT) per year for the BNB is %
about 2 x 10to 20 POT.
The rates expected per year when running in neutrino mode for 1 ton of water %
(the approximate usable fiducial volume) are about 16K neutrino interactions, %
where 11K of those would be numu CC interactions. 
This spectrum peaks ideally in the region of interest at 700 MeV as shown left of Figure 2 %
and has the target rate of neutrino interactions per year.
The spectrum and rates are based on BNB flux simulated data provided by Zarko Pavlovic (FNAL)%
appropriately propagated to the SciBoone hall.

The target is made of seven cylindrical slugs for a total target length of 71.1 cm, %
or about 1.7 inelastic interaction lengths.
The beryllium target is surrounded by a magnetic focusing horn, bending and %
sign-selecting the secondary particles that emerge from the interactions in the target along
the direction pointing to the SciBooNE detector.
The focusing is produced by the toroidal magnetic field present in the air volume %
between the horn’s two coaxial conductors made of aluminum alloy. 
The horn current pulse is approximately a half-sinusoid of amplitude 174 kA, 143 s long, %
synchronized to each beam spill. 
The polarity of the horn current flow can be (and has been) switched, in order to %
focus negatively charged mesons, and therefore produce an antineutrino instead of a neutrino beam.
The beam of focused, secondary mesons emerging from the target/horn region is further collimated, %
via passive shielding, and allowed to decay into neutrinos in a cylindrical decay %
region filled with air at atmospheric pressure, 50 m long and 90 cm in radius. 
A beam absorber located at the end of the decay region stops the hadronic and muonic %
component of the beam, and only a pure neutrino beam pointing toward the detector remains, %
mostly from a pion to mu+ nuofmu decays.



\section{Resistive Wall Monitor}
A resistive wall monitor measures the image charge that flows along the vacuum %
chamber following the beam. The image charge has equal magnitude but opposite sign.
Depending on the beam velocity, the image charge will lag behind and be spread out along its path. 
The ultimate bandwidth of such a detector is limited by this spreading of %
the electric field lines between the beam and the inside walls of the beam pipe.
The spreading angle is approximately 1/ gamma for relativistic beams ( gamma is the ratio %
of total energy to rest energy). 
The estimated bandwidth limit from spreading is 47 GHz at injection to the Main Injector %
for a 3 cm radius pipe and 8 GeV proton energy. 
In practice, the detector response is difficult to maintain above the microwave cutoff %
frequency of the beam pipe,measured to be 1.5 GHz for the elliptical beam pipe used in %
the Main Injector. 
Above cutoff, the characteristic impedance of the beam pipe and the impedance of nearby %
structures such as bellows or changes in geometry can effect accuracy.

In order to measure the image current, the beam pipe is cut and a resistive gap is %
inserted (Figure 1). 
Various ferrite cores are used to force the image current through the resistive gap rather %
than allowing it to flow through other conducting paths.
In addition to image current, other currents are often found flowing along the beam pipe. 
The gap and cores are placed inside a metal can to shunt these “noise” currents around %
rather than through the resistive gap. 
The inductance of the cores and the resistance of the gap forms a high pass filter with %
a corner frequency of R/2piL, typically a few kilohertz.
Above this frequency, cores act to minimize the net current through their center %
by inducing a current through the resistive gap that just cancels the beam current.
The gap impedance is chosen to be well below the impedance of the cores inside the shielding can. %
Several types of ferrite and microwave absorbers are used to maximize the impedance and %
minimize resonances within the desired bandwidth. 
The Main Injector shielding can has an impedance greater than 30 ohms with the ferrite cores. 
In parallel with the 1 ohm gap impedance, 30 ohms can cause frequency dependent errors %
of pm1.5\% or 0.15 db.
If the charge density around the circumference of the gap is not uniform, the voltage %
across the gap will vary around the circumference. 
The gap will act as an azimuthal transmission line transporting charge until the %
voltage equalizes. 
The time domainresponse of the detector would be distorted during this time. 
Position detectors have been made by exploiting this effect. 
The elliptical shape of the Main Injector pipe aggravates this problem (Figure 2). 
To overcome this problem, a round geometry is used for the gap and the signals from %
several monitor points equally spaced around the circumference are combined to form a single output.


