\clearpage
\pagestyle{plain}
\chapter{Overview}
This thesis is placed within the context of the %
The Accelerator Neutrino Neutron Interaction Experiment (ANNIE), a water Cherenkov %
detector built at the Fermi National Accelerator Laboratory.
The main aim of ANNIE is to study in depth the nature of neutrino-nucleus interactions %
by analysing the yield of final state neutrons produced in this kind of interactions.
New technologies, such as gadolinium-loaded water and, for the first time, %
Large Area Picosecond Photodetectors (LAPPDs) will be also employed in future phases of %
the experiment.
The measurement will have relevant implications for next generation Water Cherenkov detectors, %
in that these techniques may play a significant role %
in reducing backgrounds in relation to proton decay measurements, %
supernova neutrino observations and neutrino interaction physics.

This paper deals with the Phase I of the experiment, in which photomultipliers (PMT) are adopted %
within the tank and the forward Veto and the Muon Range Detector (MRD) are %
partially being employed.
A small container, called Neutron Volume Capture (NVC) %
is also inside the tank, for preliminary neutron yield studies.
%The MRD is a crucial component of the experiment in that it will help to tag signals properly.
The main work of the thesis consists of the development of the CAMAC electronics %
Data Acquisition system (DAQ); the already existing VME electronics DAQ, %
for the water PMTs, is also described.
Furthermore, early stage data are analysed and studied, and event reconstruction %
techniques are proposed.

All the described activities were undertaken in the Particle Physics Research Group of the %
Queen Mary University of London.

\vspace{25mm}
This thesis is divided as follows:
\begin{description}
  \item[Chapter~\ref{cha:1}] Introduction to neutrino Physics and their detection.
  \item[Chapter~\ref{cha:2}] ANNIE Experiment description.
  \item[Chapter~\ref{cha:3}] DAQ framework description and development.
  \item[Chapter~\ref{cha:4}] Data analysis and event reconstruction considerations.
  \item[Chapter~\ref{cha:5}] Measurement and main results.
  \item[Chapter~\ref{cha:6}] Conclusion.
\end{description}
