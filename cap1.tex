%%%%%%%%%%%%%%%%%%%%%%%%%%%%%%%		CHAP 1		%%%%%%%%%%%%%%%%%%%%%%%%%%%%%%%

\chapter{Introduction}
\label{cha:1}

\textcolor{blue}{Just copied and pasted random stuff, eventually i'll have to clean this mess.
Ignore it for now.}
\textcolor{red}{Brief introduction here on SM and neutrinos maybe?}



The Standard Model (SM) describes the strong, electromagnetic, and weak interactions %
of elementary particles in the framework of quantum field theory.
It is a gauge theory based on the local symmetry group 
\begin{equation}
  \label{eq:smgroup}
  \mathrm{SU(3)}_C \otimes \mathrm{SU(2)}_L \otimes \mathrm{U(1)}_Y
\end{equation}
where $C$, $L$ and $Y$ denote color, left-handed chirality and weak hypercharge, respectively.
The gauge group uniquely determines the interactions and the number of %
vector gauge bosons that correspond to the generators of the group.
They are eight massless gluons that mediate stron interactions, %
corresponding to the eight generators of SU(3)$_C$, and four gauge bosons, %
of which three are massive ($W^\pm$ and $Z$) and one is massless, corresponding %
to the three generators of SU(2)$_L$ and one generator of U(1)$_Y$, responsible for %
electroweak interactions.
%%REVIEW HERE
The electroweak part of the SM, based on the symmetry group %
$\mathrm{SU(2)}_L \otimes \mathrm{U(1)}_Y$, determines the interactions of neutrinos.
In the SM, electroweak interactions can be studied separately from strong interactions, %
because the symmetry under the color group is unbroken and there is no mixing %
between the SU(3)$_C$ and the $\mathrm{SU(2)}_L \otimes \mathrm{U(1)}_Y$ sectors.
Electromagnetic and weak interactions must be treated together because there can %
be a mixing between the neutral gauge bosons of SU(2) L and U(1) Y [543].
As in all gauge theories, the symmetry group of the SM fixes the interactions, %
i.e. the number and properties of the vector gauge bosons, with only three indepen- %
dent unknown parameters, the three coupling constants of the SU(3) C , SU(2) L and %
U(1) Y groups, all of which must be determined from experiments. On the other %
hand, the number and properties of scalar bosons and fermions are left uncon- %
strained, except for the fact that they must transform in a definite way under the %
symmetry group, i.e. they must belong to the representations of the symmetry %
group, and the fermion representations must lead to the cancellation of quantum
anomalies [31, 201, 268, 572, 517]. In the SM, the number of scalar bosons and
fermions and their arrangement in the representations of the symmetry group are
chosen in a heuristic way. The scalar bosons are chosen in order to implement, in
a minimal way, the Higgs mechanism for the generation of masses, whereas the
number and properties of fermions are determined by experiments. It is remarkable
that all the known elementary fermions can be accommodated in appropriate rep-
resentations of the symmetry group of the SM with exact cancellation of quantum
anomalies.
A puzzling feature of Nature is the existence of three generations of fermions
with identical properties, except for different masses. This feature is unexplained
in the SM.
The known elementary fermions are divided in two categories, quarks and
leptons 12 , according to the scheme
1 st generation
quarks:
leptons:
2 nd generation
3 rd generation
u (up),
c (charm),
t (top),
d (down),
s (strange),
b (bottom);
 e (electron neutrino),  (muon neutrino),  (tau neutrino),
e (electron),
 (muon),
 (tau).
They are distinguished by the fact that quarks participate in all the interactions
(strong, electromagnetic, weak, and gravitational), whereas leptons participate in
all the interactions except strong interactions. The masses and electric charges of
quarks and leptons are given in Tables 3.1 and 3.2. The corresponding antiparticles
have the same mass and opposite electric charge. All fundamental fermions have
spin 1/2.

A neutrino is a chargless lepton, therefore it is a fermion which interacts only %
via the weakforce.

The predictions of the existence of the W and Z bosons, the gluon, %
the top and charm quarks made the fortune of the Standard Model (SM).
Their redicted properties were experimentally confirmed with good precision and %
the recent discovery of the Higgs Boson~\cite{atlas:higgs}~\cite{cms:higgs} %
is the last crowning achievement of SM.

Despite being the most successful theory of particle physics to date, %
the Standard Model is actually limited in its approximation to reality, in that %
some clear evidences cannot be explained.
The most outstanding discovery is the neutrino oscillations, %
which has proved that the neutrinos are not all massless, %
as it is assumed by theory.
Mass terms for the neutrinos can be included in the SM, %
with the implications of theoretical problems.
Likewise, the SM is unable to provide an explanation of the observed asymmetry %
between matter and anti-matter.
It was noted by Sakharov~\cite{sak:cp} that a solution to this puzzle would require %
some form of C and CP violation in the early Universe, along with Baryon number violation %
and out-of-equilibrium interactions.
These facts suggest that the Standard Model is not a complete theory and %
additional physics Beyond the Standard Model (BSM) is required.

Even though evidence of CP-violation has been found in hadron physics %
!danielscullyintroduction!, this does not suffice for the asymmetry required to %
explain cosmological observations.
M. Fukugita and T. Yaganida pointed out an elegant mechanism to generate the baryon excess %
from lepton asymmetries.
This process is referred as Leptogenesis and necessitates a Standard Model extension, %
such as the introduction of right-handed neutrinos, permitting implementation of %
the see-saw mechanism and providing the neutrinos with mass.
On the other hand, the extended model is able to spontaneously generate leptons %
from the decays of right-handed neutrinos.
Eventually, the lepton asymmetry is converted to baryon asymmetry by sphalerons.
This process is referred as Leptogenesis and involves Majorana mass terms and the %
seesaw may transform into the baryon number excess through the
unsuppressed baryon number wolatlon of electroweak processes at high temperatures




 Once again the discovery of neutrino oscillations could hold the answer since the phenomenon also allows for the possibility of leptonic CP-violation. Various mechanisms have been proposed that would allow CP-violation in the lepton sector to provide the necessary asymmetry in the baryons [5]. Establishing whether or not CP-violation does occur in neutrinos is therefore a priority.

 It is clear that the study of neutrinos is of vital importance to the future development of particle physics, in particular through the study of oscillations. However, as will be discussed in Chapter 2, precision measurements of neutrino oscillations are predicated on improving our understanding of neutrino-nucleus interactions.

 There are multiple channels through which neutrinos can interact with nuclei and, as described in Chapter 3, many of them are in need of improvements in both our experimental and theoretical understanding. In Chapter 4 one interaction channel, coherent pion production, is identified as being of particular importance to the field, in need of experimental input, and with the potential for significant improvements in the near future. The T2K neutrino oscillation experiment, described in Chapter 5, provided an opportunity to contribute to this field, the results of which are reported in Chapter 6. 

The neutrino was postulated first by Wolfgang Pauli in 1930 to explain %
how beta decay could conserve energy, momentum, and angular momentum (spin).
In contrast to Niels Bohr, who proposed a statistical version of the conservation %
laws to explain the event, Pauli hypothesized an undetected particle that he %
called a ``neutron'', using the same -on ending employed for naming both %
the proton and the electron.
He considered that the new particle was emitted from the nucleus together with %
the electron or beta particle in the process of beta decay.[8][nb 2]

In Fermi's theory of beta decay, Chadwick's large neutral particle could decay to a proton, %
electron, and the smaller neutral particle (flavored as an electron antineutrino):
%        n0 → p+ + e− + νe

Fermi's paper, written in 1934, unified Pauli's neutrino with %
Paul Dirac's positron and Werner Heisenberg's neutron–proton model and gave a %
solid theoretical basis for future experimental work.
However, the journal Nature rejected Fermi's paper, saying that the %
theory was ``too remote from reality''.
He submitted the paper to an Italian journal, which accepted it, %
but the general lack of interest in his theory at that early date caused him to %
switch to experimental physics.[10]:24[11]

Nevertheless, even in 1934 there were hints that Bohr's idea — %
that the energy conservation laws were not followed — was incorrect.
At the Solvay conference of 1934, the first measurements of the energy %
spectra of beta decay were reported, and these spectra were found to impose a %
strict limit on the energy of electrons from each type of beta decay.
Such a limit was not expected if the conservation of energy was not upheld, %
in which case any amount of energy would be expected to be statistically available %
in at least a few decays.
The natural explanation of the beta decay spectrum as first measured in 1934 %
was that only a limited (and conserved) amount of energy was available, %
and a new particle was sometimes taking a varying fraction of this limited energy, %
leaving the rest for the beta particle.
Pauli made use of the occasion to publicly emphasize that the still-undetected %
``neutrino'' must be an actual particle.

The idea of neutrino was put forward by W. Pauli in 1930.
This was a dramatic time in physics.
After it was established in the Ellis and Wooster experiment that %
the average energy of the electrons produced in the beta-decay is significantly smaller %
than the total released energy, only the existence of neutrino, a neutral particle with %
a small mass and a large penetration length (much larger then the penetration length %
of the photon), which is emitted in the $\beta$-decay together with the electron, could %
save the fundamental law of the conservation of energy. 
At the time when the neutrino hypothesis was proposed the only known elementary %
particles were electron and proton. In this sense neutrino (more exactly electron %
neutrino) is one of the “oldest” elementary particles.
However, the existence of the %
neutrino was established only in the middle of the fifties when neutron, muon, pions, %
kaons, $\Lambda$ and other strange particles were discovered.
We know at present that the twelve fundamental fermions exist in nature: six %
quarks u, d, c, s, t, b, three charged leptons e, $\mu$, $\tau$ and three neutrinos $\nu_e$, %
$\nu_\mu$, $\nu_\tau$.
They are grouped in the three families, which differ in masses of particles but have %
universal electroweak interaction with photons and vector W$^\pm$, Z bosons.
In the Lagrangian of the electroweak interaction, neutrinos enter on the same footings %
as the quarks and charged leptons.
In spite of this similarity of the electroweak %
interaction neutrinos are special particles. 


\section{Neutrino Physics}
\textcolor{red}{Neutrino physics, like wak interactions.}
\label{sec:1.1}
There are two basic differences between neutrinos and other fundamental fermions.
\begin{itemize}
  \item At all available at present energies, cross section of the interaction of neutrinos %
    with matter is many order of magnitude smaller than the cross section of the %
    electromagnetic interaction of leptons with matter (via the exchange of the virtual %
    photon).
    This is connected with the fact that neutrinos interact with matter via %
    the exchange of the heavy virtual W$^\pm$ and Z bosons.
  \item Neutrino masses are many order of the magnitude smaller than the masses of %
    leptons and quarks.
    Because of the extreme smallness of the neutrino cross section, special methods %
    of the detection of neutrino processes must be developed.
\end{itemize}

The Physics of neutrino-nucleus interactions is strictly influenced by the complex interplay %
of multiple particles and current models are limited in their approximation to reality.
As more neutrino data become available, lack of knowledge of the fine details of the %
interactions begins to narrow the reach of future experiments.
Under this circumstances, experiments that observe neutrino-nucleus interactions using %
different techniques are essential to develop more accurate models that allow an era of %
precision physics in neutrino experiments.
A key handle in understanding neutrino-nucleus interaction is the number and type of nucleons
ejected from the interaction, which are a valuable constraint on theoretical models.

One example is the first double differential cross section for charged-current quasi-elastic %
(CCQE) interactions, published by the MiniBooNE experiment [ref].
It has been shown that these and other data are finely described by models including %
two-body currents, where low-energy neutrinos scatter off correlated pairs of nucleons.
A relevant predicted consequence of two-body currents is higher multiplicity of %
final-state nucleons, indeed.

\section{Neutrino detection}
\label{sec:1.3}
LAr?

\subsection{Water cherenkov}
\label{sec:1.3.1}
As a charged particle travels, it disrupts the local electromagnetic field in its medium.
In particular, the medium becomes electrically polarized by the particle's electric field.
If the particle travels slowly then the disturbance elastically relaxes back to %
mechanical equilibrium as the particle passes.
When the particle is traveling fast enough, however, the limited response speed of %
the medium means that a disturbance is left in the wake of the particle, %
and the energy contained in this disturbance radiates as a coherent shockwave.

``Ring-imaging'' Cherenkov detectors take advantage of a phenomenon called Cherenkov light.
Cherenkov radiation is produced whenever charged particles such as electrons or muons %
are moving through a given detector medium somewhat faster than the speed of light in that medium.
In a Cherenkov detector, a large volume of clear material such as water or ice %
is surrounded by light-sensitive photomultiplier tubes.
A charged lepton produced with sufficient energy and moving through %
such a detector does travel somewhat faster than the speed of light in the detector medium %
(although somewhat slower than the speed of light in a vacuum).
The charged lepton generates a visible ``optical shockwave'' of Cherenkov radiation.
This radiation is detected by the photomultiplier tubes and shows up as a characteristic %
ring-like pattern of activity in the array of photomultiplier tubes.
As neutrinos can interact with atomic nuclei to produce charged leptons which emit %
Cherenkov radiation, this pattern can be used to infer direction, energy, %
and (sometimes) flavor information about incident neutrinos.

\begin{equation}
  \label{eq:ch_Eth}
  \frac{E}{m_0 c^2} > \frac{1}{\sqrt{1-1/n^2}}
\end{equation}

\subsection{Neutron detection}
\label{sec:1.3.2}
The Super-Kamiokande (SK) [1] is the largest light water Cherenkov detector %
that has been successfully observing solar, atmospheric and accelerator neutrinos.
Recently the addition of 0.2\,\% of a water soluble gadolinium (Gd) compound to SK %
has been proposed [2].
This modification can greatly improve the detection sensitivity of anti-electron neutrinos.
The inverse beta interaction in the water, emits a positron and a neutron.
The positron, radiating Cherenkov photons, is immediately detected.
The neutron is quickly thermalized in the water, and is then captured by Gd with a %
probability of 90\,\%.
Upon capturing a neutron the Gd emits 3-4 gamma rays having a total energy of %
about 8 MeV.
The time and spatial correlation of the positron and neutron capture events (20 us and 4 cm) %
can significantly reduce the backgrounds, and hence enhance the nu e signal events.


