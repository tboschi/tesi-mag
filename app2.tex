%%%%%%%%%%%%%%%%%%%%%%%%%%%%%%%		APP		%%%%%%%%%%%%%%%%%%%%%%%%%%%%%%%

\chapter{Monte Carlo simulation}
\label{app:B}


 A simple Monte Carlo simulation was developed in order to help study the Cherenkov radiation production %
 to the passage of a charged particle.
 The minimal model to study the distribution of the photon on the floor of the tank and the %
 role of the walls is implemented.
 This considers only vertical muons with $\beta = 1$, and both scattering with nucleons and the related %
 energy loss are ignored.
 A uniform Mersenne Twister (MT) generator\footnote{The Mersenne Twister is a pseudorandom number %
   generator (PRNG) and its name is due to the fact that its period length is a Mersenne prime.
   The version used employes the Mersenne prime $2^{219937}-1$.} defines the position of the %
 muon entering in the tank.
 It propagation is stopped every \np{0.1}~cm and a handful of photons is generated isotropically, %
 using the same PRNG.
 The number of photons at each step is fixed a priori and Eq.~\ref{eq:numphoton} gives:
 \begin{equation}
   N_\gamma \simeq 27\,,
 \end{equation}
 with $L = \np{0.1}$~cm, $\theta_C$ maximum, and $\lambda$ interval from 300 to 500~nm.
 As soon as all the photons ends their route, the muon is moved and the process is repeated again, until %
 even the charged lepton hits the bottom.

 The track of the gammas is precisely defined by its origin, the direction it is emitted, and %
 the reflection coefficient of the PVC walls, because they travel in straight lines.
 With elementary geometry, it is easy to calculate the track of the photons.
 Since the muon travels at ultrarelativistic speed, the wavefront of the photons has the maximum %
 angle allowed by the Cherenkov effect, which in water is $\theta_C = 41^\circ$.
 The muon and the photons tracks are exemplified in Fig.~\ref{fig:muontrack}.
 The impact of the walls is managed by the same MT generator.

 For every reflection coefficient, hundred muons are simulated and some illustrative results are shown %
 in Fig.~\ref{fig:pattern}.
 Being the tank 4~metres tall, the muon makes 4000 steps, hence \np{108000} photons are %
 generated everytime.
 Obviously only when the walls entirely reflects (100\,\% reflection), all the photons reach the bottom.
 Despite its simplicity, this model is capable of producing interesting results.

 %\begin{wrapfigure}{R}{0.5\textwidth}
 \begin{figure}
  \centering
  \def\svgwidth{0.7\textwidth}
  \import{pics/}{muontrack.tex}
  \caption{Track of the incoming muon (in purple) and the emitted photons (in green).
    On the left, the side view underlines the angle of the wavefront with respect to the z-axis, %
    $\theta_C$.
    On the rigth, the top view shows the xy-plane reflections of the photon.
    Since $\theta$ and $\alpha$ are known, the angle after the reflection is %
    $\beta = \pi - \theta+2\alpha$.}
  \label{fig:muontrack}
 %\end{wrapfigure}
 \end{figure}

\begin{figure}
  \centering
  \subfloat[Muon at the centre of the cylinder and 100\,\% wall reflection.]%
    {\includegraphics[width=0.48\textwidth]{pics/mcentre_ref100.pdf}} \hfill 
  \subfloat[Muon at the centre of the cylinder and 0\,\% wall reflection.]%
    {\includegraphics[width=0.48\textwidth]{pics/mcentre_ref000.pdf}} \\
  \subfloat[Muon at a distance greater than 120~cm from the centre and 100\,\% wall reflection.]%
    {\includegraphics[width=0.48\textwidth]{pics/mlateral_ref100.pdf}} \hfill 
  \subfloat[Muon at a dsitance greater than 120~cm from the centre and 0\,\% wall reflection.]%
    {\includegraphics[width=0.48\textwidth]{pics/mlateral_ref000.pdf}} 
    \caption{Patternc createdhotons deposited on the bottom of the water tank.}
  \label{fig:pattern}
\end{figure}
