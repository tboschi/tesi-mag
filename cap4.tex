%%%%%%%%%%%%%%%%%%%%%%%%%%%%%%%		CHAP 4		%%%%%%%%%%%%%%%%%%%%%%%%%%%%%%%

\chapter{Data Analysis procedures}
\label{cha:4}
This chapter deals with the data analysis techniques developed in order to study PMTs data.
These methods focus on the rejection of background signals with respect to event signals which is %
done with inidivual pulse analysis.
Some of these procedures could be used in future pahses of the experiment, in particular %
the ones with LAPPDs.

Specification of runs
As explained in chapter~\ref{cha:3}, the DAQ creates a new \emph{Run} everytime it is restarted.
Due to being an R\&D phase, the DAQ has been improved in various occasion during data taking, %
hence the size and the number of files of the \emph{Runs} are not constant.
Every single file is a ROOT file containing a Tree and one of its branches helds the post-processed %
buffer from the digitisers.
The data array is scanned by the Analysis software, entry by entry.

\textcolor{red}{How measurement are made.}

\section{Data structure}
The Main DAQ creates ROOT file containing the PMT data collected (see section~ref).
The buffer retrieved from the VME bards is properly split for each trigger signal, in that each %
set of data consists of 80~$\mu$s worth of digitised signal.
Each file holds 383 full buffers and this translates to 1532 trigger: the effective time held in %
a single file is \np{122.56}~ms.
The Analysis software scans all the files which the \emph{Run} consists of.
The data array is the time profile of the PMTs outputs.

\textcolor{red}{Picture of full buffer and post processed}


\section{Inidivual pulse analysis}

The code looks for any peak above a certain voltage threshold, defining a time window %
around it, of a predefined length.
A length of 100 samples was chosen, resulting in a \np{0.8}~$\mu$s long window.
The position of the peak inside this window was set to 20\% of its length.
This set of data is called ``pulse'' and it's afterward post processed
The length and the peak position were choosen in a way that the reflection peaks are counted as %
one single pulse.

Every pulse is analysed by the software and some quantities are infered from each.
These values are:

%\begin{center}
%\begin{varwidth}{\textwidth}
\begin{itemize}
%  \item[Length] of the pulse array;
%  \item[Trigger] number and PMT ID;
  \item[\bfseries Baseline] is given by the average of first 10 points of the pulse; %
    this value is then subtracted from the whole array.
  \item[\bfseries Height] is the height of the peak (maximum) with respect to the zero.
  \item[\bfseries Peak to valley] it is the time distance from the peak to the valley (minimum);
  \item[\bfseries Start] is position of the 25\% of the rising edge of the pulse, estimated with %
    precision using cubic interpolation\footnote{The algorithm looks for four points around the %
    threshold which are used to define a cubic function. Using Newton's method, the time position %
    is found.}.
  \item[\bfseries Width] is the time that spans from Time to the 5\% of the falling edge %
    of the pulse, calculated using the same algorithm employed for Time.
  \item[\bfseries Charge] is the sum of the 5 points around peak, weighted by the bin width.
  \item[\bfseries Energy] is the sum of the points form Time to Time+Width, with bin weight.
  \item[\bfseries Area] is the area of the modulus of the pulse.
  \item[\bfseries Time of flight] is the time difference between Time and RWM signal.
  \item[\bfseries Previous] is the Time of the previous pulse, if from the same PMT.
%  \item[If] VETO or both MRD layers were on;
%  \item[The] whole pulse array;
\end{itemize}
%\end{varwidth}
%\end{center}

\begin{figure}
  \centering
  \def\svgwidth{0.45\textwidth}
  \subfloat[To be fixed.]{\import{pics/}{timepoints.pdf_tex}} \qquad
  \def\svgwidth{0.45\textwidth}
  \subfloat[To be fixed.]{\import{pics/}{areapoints.pdf_tex}}
  \caption{Explanation of the pulse analysis parameters.}
\end{figure}

\textcolor{red}{Moreover, VETO and MRD layers coincidences are recorded.}

\subsection{Event definition}

For every Trigger, the time distribution of the peaks is also studied, since it helps to find %
coincidences between PMTs.
With the help of an histogram, an allowance of $\np{0,8}~\mu$s is given to count the coincidence %
magnitude of the Event and an \emph{Event} is appointed to any cluster delimited by a bin %
greater than a defined threshold up to the number of PMTs fired at the same time falls to zero.
[insert event picture here]
In order to estimate the precise time position of the event, a weighted average is done among bins.

\textcolor{red}{Improve here!}
{\color{blue}
What can be counted as an event?
Cosmic: it would be great to calculate the expected flux in the tank. Here?
Beam: the pmts will only takes a snippet from the cherenkov cone. So?
Radiation: this sucks.
I could estimate the number of PMTs on for each scenarios.
}
\textcolor{red}{Place some pictures here and there.}


\subsection{Signal and noise discrimination}
I mean event from dark noise. No dark noise run was taken (empty tank, maybe when LAPPDs will be %
installed).
A time window is defined using the Event time position and it is used as a rejection interval.
Every pulse that falls within this interval is designated as a ``signal''; otherwise it is ``noise''.
Following this discrimination paradigm, two ROOT trees, one for signals and one for events, %
are filled with the pulses features, along with the pulse shape itself.
In ROOT file two trees are saved, tEvent and tNoise

\section{Run selection}

For the analysis, three couples of runs were selected:
\begin{itemize}
  \item[\bfseries R93/R94] beam off, trigger random	
  \item[\bfseries R120/R121] after card synchronisation	
  \item[\bfseries R145/R146] PMT mounted on NCV		
\end{itemize}

\begin{center}
  \small
  \begin{tabular}{crcrr}
    \toprule
    \textbf{Run} & \textbf{Data size}	& N of files	& N of Triggers	& Total time	\\
    		 &	(MB)		&		&		& (ms)	\\
    \midrule
      93		 & $ \np{65027.961} $	& 88	& \np{134816}	& \np{10785.280}	\\
      94		 & $ \np{41102.486} $	& 54	& \np{82728}	& \np{6618.240}	\\
      120		 & $ \np{19887.893} $	& 43	& \np{65876}	& \np{5270.080}	\\
      121		 & $ \np{6735.270}  $	& 15	& \np{22980}	& \np{1838.400}	\\
      145		 & $ \np{37204.525} $	& 80	& \np{122560}	& \np{9804.800}	\\
      146		 & $ \np{39050.909} $	& 84	& \np{128688}	& \np{10295.040}	\\
    \bottomrule
  \end{tabular}
\end{center}
